\begin{longtable}[]{@{}ll@{}}
\caption{Patterns (principes) GRASP}\label{tableGRASP}\tabularnewline
\toprule
\begin{minipage}[b]{0.15\columnwidth}\raggedright
Pattern\strut
\end{minipage} & \begin{minipage}[b]{0.77\columnwidth}\raggedright
Description\strut
\end{minipage}\tabularnewline
\midrule
\endfirsthead
\toprule
\begin{minipage}[b]{0.15\columnwidth}\raggedright
Pattern\strut
\end{minipage} & \begin{minipage}[b]{0.77\columnwidth}\raggedright
Description\strut
\end{minipage}\tabularnewline
\midrule
\endhead
\begin{minipage}[t]{0.15\columnwidth}\raggedright
Expert en information\label{tab_GRASPExpert}\strut
\end{minipage} & \begin{minipage}[t]{0.77\columnwidth}\raggedright
Un principe g\'en\'eral de conception d'objets et d'affectation des
responsabilit\'es.\newline
\newline
Affecter une responsabilit\'e \`a l'expert -- la classe qui poss\`ede les
informations n\'ecessaires pour s'en acquitter.\strut
\end{minipage}\tabularnewline
\hline
\begin{minipage}[t]{0.15\columnwidth}\raggedright
Cr\'eateur\label{tab_GRASPCreateur}\strut
\end{minipage} & \begin{minipage}[t]{0.77\columnwidth}\raggedright
Qui cr\'ee~? (Notez que Fabrique Concr\`ete est une solution de rechange
courante.)\newline
\newline
Affectez \`a la classe B la responsabilit\'e de cr\'eer une instance de la
classe A si l'une des assertions suivantes est vraie:

\begin{enumerate}
\def\labelenumi{\arabic{enumi}.}
\tightlist
\item
  B contient A
\item
  B agr\`ege A
\item
  B a les donn\'ees pour initialiser A
\item
  B enregistre A
\item
  B utilise \'etroitement A
\end{enumerate}\strut
\end{minipage}\tabularnewline
\hline
\begin{minipage}[t]{0.15\columnwidth}\raggedright
Contr\^oleur\label{tab_GRASPControleur}\strut
\end{minipage} & \begin{minipage}[t]{0.77\columnwidth}\raggedright
Quel est le premier objet en dehors de la couche pr\'{e}sentation qui re\c{c}oit
et coordonne (\guillemotleft~contr\^ole~\guillemotright) les op\'erations syst\`eme~?\newline
\newline
Affectez une responsabilit\'e \`a la classe qui correspond \`a l'une de ces
d\'efinitions :

\begin{enumerate}
\def\labelenumi{\arabic{enumi}.}
\tightlist
\item
  Elle repr\'esente le syst\`eme global, un \guillemotleft~objet racine~\guillemotright, un \'equipement
  ou un sous-syst\`eme (\emph{contr\^oleur de fa\c{c}ade}).
\item
  Elle repr\'esente un sc\'enario de cas d'utilisation dans lequel
  l'op\'eration syst\`eme se produit (\emph{contr\^oleur de session} ou
  \emph{contr\^oleur de cas d'utilisation}). On la nomme \emph{GestionnaireX} où \emph{X} est le nom du cas d'utilisation.
\end{enumerate}\strut
\end{minipage}\tabularnewline

\hline
\begin{minipage}[t]{0.15\columnwidth}\raggedright
  Faible Couplage (évaluation)\strut
  \end{minipage} & \begin{minipage}[t]{0.77\columnwidth}\raggedright
  Comment minimiser les dépendances?\newline
  \newline
  Affectez les responsabilités de sorte que le couplage (inutile) demeure faible. Employez ce principe pour évaluer les alternatives.\strut
  \end{minipage}\tabularnewline

  \hline
  \begin{minipage}[t]{0.15\columnwidth}\raggedright
    Forte Cohésion (évaluation)\strut
    \end{minipage} & \begin{minipage}[t]{0.77\columnwidth}\raggedright
      Comment conserver les objets cohésifs, compréhensibles, gérables et, en conséquence, obtenir un Faible Couplage~?\newline
    \newline
    Affectez les responsabilités de sorte que les classes demeurent cohésives. Employez ce principe pour évaluer les différentes solutions.\strut
    \end{minipage}\tabularnewline
  
    \hline
    \begin{minipage}[t]{0.15\columnwidth}\raggedright
      Polymorphisme\strut
      \end{minipage} & \begin{minipage}[t]{0.77\columnwidth}\raggedright
        Qui est responsable quand le comportement varie selon le type~?\newline
      \newline
      Lorsqu'un comportement varie selon le type (classe), affectez la responsabilité de ce comportement - avec des opérations polymorphes - aux types pour lesquels le comportement varie.

\strut
      \end{minipage}\tabularnewline
    
      \hline
      \begin{minipage}[t]{0.15\columnwidth}\raggedright
        Fabrication Pure\strut
        \end{minipage} & \begin{minipage}[t]{0.77\columnwidth}\raggedright
          En cas de situation désespérée, que faire quand vous ne voulez pas transgresser les principes de faible couplage et de forte cohésion~?\newline
        \newline
        Affectez un ensemble très cohésif de responsabilités à une classe \guillemotleft~comportementale~\guillemotright\  artificielle qui ne représente pas un concept du domaine - une entité fabriquée pour augmenter la cohésion, diminuer le couplage et faciliter la réutilisation.
\strut
        \end{minipage}\tabularnewline
      
        \hline
        \begin{minipage}[t]{0.15\columnwidth}\raggedright
          Indirection\strut
          \end{minipage} & \begin{minipage}[t]{0.77\columnwidth}\raggedright
            Comment affecter les responsabilités pour éviter le couplage direct~?\newline
            \newline
            Affectez la responsabilité à un objet qui sert d'intermédiaire avec les autres composants ou services.\strut
          \end{minipage}\tabularnewline
        
          \hline
          \begin{minipage}[t]{0.15\columnwidth}\raggedright
            Protection des variations
\strut
            \end{minipage} & \begin{minipage}[t]{0.77\columnwidth}\raggedright
              Comment affecter les responsabilités aux objets, sous-systèmes et systèmes de sorte que les variations ou l'instabilité de ces éléments n'aient pas d'impact négatif sur les autres~?\newline
            \newline
            Identifiez les points de variation ou d'instabilité prévisibles et affectez les responsabilités afin de créer une \guillemotleft~interface~\guillemotright\ stable autour d'eux.
\strut
            \end{minipage}\tabularnewline
          
  \bottomrule
\end{longtable}
